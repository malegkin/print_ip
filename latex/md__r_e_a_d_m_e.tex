Реализовать функцию печати условного ip-\/адреса. Адрес может быть представлен в виде произвольного целочисленного типа. Представление не зависит от знака типа. Выводить побайтово начиная со старшего с символом . в качестве разделителя.

Адрес может быть представлен в виде строки. Выводится как есть.

Адрес может быть представлен в виде контейнеров std\-::list, std\-::vector.

Выводится содержимое контейнера поэлементно и разделяется '.'

Дополнительно адрес может быть представлен в виде std\-::tuple при условии, что все типы одинаковы. Выводится содержимое поэлементно и разделяется '.'. Опционально.

Прикладной код должен содержать следующие вызовы\-:
\begin{DoxyItemize}
\item Печать адреса как char(-\/1)
\item Печать адреса как short(0)
\item Печать адреса как int(2130706433)
\item Печать адреса как long(8875824491850138409)
\item Печать адреса как std\-::string
\item Печать адреса как std\-::vector
\item Печать адреса как std\-::list
\item Опционально. Печать адреса как std\-::tuple
\end{DoxyItemize}

Добавить в .travis.\-yml на этапе сборки вызов doxygen и публикацию html версии документации на github-\/pages. Подробное описание на странице\-: \mbox{[}\href{https://docs.travis-ci.com/user/deployment/pages/}{\tt https\-://docs.\-travis-\/ci.\-com/user/deployment/pages/}\mbox{]}

Включить в репозиторий файл {\ttfamily Doxyfile} с включенными опциями {\ttfamily H\-A\-V\-E\-\_\-\-D\-O\-T} и {\ttfamily E\-X\-T\-R\-A\-C\-T\-\_\-\-A\-L\-L}.

\section*{Требования к реализации}


\begin{DoxyItemize}
\item Бинарный файл и пакет должны называться print\-\_\-ip.
\item Результат опубликовать в своём репозитории на bintray.
\item Функция печати должна быть одной шаблонной функцией с частичной специализацией.
\item Специализация для целочисленного представления должна быть единственная.
\item Специализация для контейнеров должна быть единственная.
\end{DoxyItemize}

\section*{Проверка}

Задание считается выполненным успешно, если после просмотра кода, подключения репозитория, установки пакета и запуска бинарного файла командой\-: {\ttfamily  print\-\_\-ip} будут выведены адреса\-: ``` 255 0.\-0 127.\-0.\-0.\-1 123.\-45.\-67.\-89.\-101.\-112.\-131.\-41 ``` вслед за которыми будут выведены адреса из строки, контейнеров и опционально из кортежа. 